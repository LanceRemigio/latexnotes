
\section{Double Summations and Products}

We discovered in an earlier section that given any doubly indexed array of real numbers \( \{  a_{ij} : i,j \in \N  \}\), it can be an ambiguous task to define 
\[ \sum_{i,j = 1}^{\infty} a_{ij}. \tag{1}\]
We also observed that performing \textit{iterated summations}can lead to different summations. Of course, this can be avoided completely if we were to define the partial sum of (1) in the following way 
\[ s_{mn} =  \sum_{i=1}^{m} \sum_{j=1}^{n} a_{ij}\]
for \( m, n \in \N \). In order for the sum of (1) to converge we have to have the following hold:
\[ \sum_{i,j = 1 }^{\infty} a_{ij} = \lim_{n \to \infty} s_{mn}\]

\subsubsection{Exercise 2.8.1} Using the particular array \( (a_{ij})\) from Section 2.1, compute \( \lim_{n \to \infty} s_{mn}\). How does this value compare to the two iterated values for the sum already computed?  
\begin{proof}

\end{proof}



