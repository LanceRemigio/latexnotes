\section{The Algebraic and Order Limit Theorems}

The goal of having a rigorous definition of convergence in Analysis is to prove statements about sequences in general like the notion of "boundedness" which we will define below.

\begin{tcolorbox}
\begin{defn}
A sequence \( (x_n) \) is \textit{bounded} if there exists a number \( M > 0 \) such that \( | x_n | \leq M \) for all \( n \in \N \).
\end{defn}
\end{tcolorbox}

Geometrically, this means that we can find an interval \( [-M, M]\) that contains every term in the sequence \( (x_n)\).
This naturally leads us to the point that all convergent sequences are bounded i.e 

\begin{tcolorbox}
\begin{thm}
Every convergent sequence is bounded.
\end{thm}
\end{tcolorbox}

\begin{proof}
Assume \( (x_n) \) converges to a limit \( \ell\). This means that given \( \epsilon =  1\), we can find an 
\( N \in \N\) such that for every \( n \geq N \), we can say that 
\begin{align*}
    \implies&| x_n - \ell |  <  1  \\
    \iff &-1 < x_n - \ell < 1 \\
    \iff &\ell  - 1 < x_n < \ell + 1.
\end{align*}
Note the terms of the sequence \( (x_n)\) can be found in the open interval \( (\ell - 1, \ell + 1)\). Since \( \ell \in \R \) can either be positive or negative, we can conclude that  
\[ | x_n | < | \ell | + 1 \]
for all \( n \geq N \) where
\[ M = \max \{ | x_1 |, | x_2 |, ..., | \ell | + 1 \}.\]
Hence, it follows that \( | x_n  | \leq M \) for all \( n \in \N \) as desired.
\end{proof}

\begin{tcolorbox}
    \begin{thm}[Algebraic Limit Theorem]
        Let \( \lim a_n = a \), and \( \lim b_n = b \). Then, 
        \begin{enumerate}
            \item[(i)] \( \lim(ca_n) = ca \) for all \( c \in \R \);
            \item[(ii)] \( \lim (a_n + b_n) = a + b \);
            \item[(iii)] \( \lim (a_nb_n) = ab\);
            \item[(iv)] \( \lim (a_n / b_n) = a / b \) provided that \( a \neq 0\).
        \end{enumerate}
\end{thm}
\end{tcolorbox}

\begin{proof}[Proof of (i)]
We begin by proving part \( (i)\). Suppose \( a_n \to a \). Then for every \( \epsilon  > 0 \), there exists 
\( N \in \N \) such that for every \( n \geq N \), we have 
\[ | a_n  - a  | < \epsilon / | c |. \tag{1}\]
In order to show \( (i)\), we need to show that 
\[ | ca_n - ca | < \epsilon.\]
Hence, observe that 
\begin{align*}
| ca_n - ca |&< | c(a_n - a) | \\
&< | c | | a_n - a  | \\  
&< | c | \frac{ \epsilon  }{ | c |} \\ 
&= \epsilon.   
\end{align*}
If \( c = 0 \), then our sequence \( (ca_n)\) reduces to the sequence \( \{0,0,0,...,0 \}\) which is clearly converging to \( ca = 0 \).
Hence, we have attained our desired property that \( \lim (ca_n) = ca\).
The parts are left to you to prove.
\end{proof}
\begin{proof}[Proof of (ii)]
    To show part \( (ii)\), it suffices to show that for every \( \epsilon> 0 \), there exists \( N \in \N \) such that for every \( n \geq N \), we have 
    \[ | a_n + b_n - (a+b) | < \epsilon.\] Hence, we start with the left side of (ii). Since \( a_n \to a \) and \( b_n \to b\), there exists \( N_1, N_2 \in \N \). We can choose \( N = \max \{ N_1, N_2 \}\) such that for every \( n \geq N \), we can say that 
    \begin{align*}
     | a_n + b_n - (a + b) | &< | (a_n-a) + (b_n - b) |  \\
                             &< | a_n - a  |  + | b_n - b | \\ 
                             &< \frac{ \epsilon }{2} + \frac{ \epsilon }{ 2} \\  
                             &= \epsilon. 
    \end{align*}
Hence, it follows that \( \lim (a_n + b_n) = a + b \) as required.

\end{proof}

\begin{proof}[proof of (iii)]
    To show part \( (iii)\), it suffices to show for every \( \epsilon  > 0 \), there exists \( N \in \N \) such that for every \( n \geq N \), we have 
    \[ | a_nb_n - ab | < \epsilon.\]
Since \( a_n \to a \) and \( b_n \to b\), there exists \( N_1 , N_2 \in \N \). We can choose \( N = \max \{ N_1, N_2  \}\) such that for every \( n \geq N \), we can say that 
\begin{align*}
 | a_nb_n - ab |&< | a_nb_n -a_nb + a_nb - ab |  \\
                &< | a_n (b_n - b ) + b (a_n - a)| \\ 
                &< | a_n (b_n - b) |  + | b (a_n - a) | \\ 
                &< | a_n | | b_n - b  |  + | b | | a_n - a |  \\
                &< M \frac{ \epsilon }{2 M }  + | b | \frac{ \epsilon }{2 | b |} \tag{ \( a_n \) is bounded } \\ 
                &< \epsilon  
\end{align*}
Hence, it follows that \( \lim (a_nb_n) = ab\).
\end{proof}

\begin{proof}[Proof of (iv)]
To show part (iv), it suffices to show for every \( \epsilon  > 0 \), there exists an \( N \in \N\) such that for every \( n \geq N \), we have 
\[ \Big| \frac{a_n}{b_n} - \frac{a}{b} \Big| < \epsilon.\]
Since \( a_n \to a \) and \( b_n \to b\) with \( b \neq 0 \), there exists an \( N_1, N_2 \in \N   \) such that whenever \( n \geq N_1, N_2\), we can have
\begin{align*}
 | a_n - a  |&<  M \epsilon / 2,  \\
 | b_n - b | &<  \frac{ | b |}{ | a |} \cdot \frac{ M \epsilon }{2}.
\end{align*}



we can choose \( N = \max \{ N_1, N_2 \}\) so that 
\begin{align*}
   \Big| \frac{a_n}{b_n} - \frac{a}{b} \Big| &=  \Big| \frac{a_nb - b_n a}{b_nb} \Big|   \\
                                     &=  \Big| \frac{a_nb - b_n a}{b_nb} \Big| \\
                                     &= \Big| \frac{a_nb - ab + ab- b_n a}{b_nb} \Big| \\
                                     &=  \Big| \frac{b(a_n - a) + (b- b_n)a}{b_nb} \Big| \\
                                     &<  \frac{|a_n - a|}{|b_n|} + \frac{ | a |}{ | b |} \cdot \frac{|b_n - b|}{|b_n|} \\
                                     &< \frac{ M \epsilon }{ 2M} + \frac{ | a |}{ | b |} \cdot \frac{ | b | M  \epsilon}{  | a | 2 M} \tag{ \( b_n\) bounded} \\
                                     &= \epsilon. 
\end{align*}
Hence, it follows that \( \lim ( \frac{a_n}{b_n} ) = \frac{a}{b} \) provided that \( b \neq 0\).



\end{proof}

\begin{tcolorbox}
    \begin{thm}[Order Limit Theorem] 
    Assume \( \lim a_n  = a\) and \( \lim b_n =  b\).
    \begin{enumerate}
        \item[(i)] If \( a_n \geq 0 \) for all \( n \in \N \), then \( a \geq 0\).
        \item[(ii)] If \( a_n \leq b_n\) for all \( n \in \N \), then \( a \leq b \).
        \item[(iv)] If there exists \( c \in \R \) for which \( c \leq b_n\), for all \( n \in \N \), then 
        \( c \leq b \). Similarly, if \( a_n \leq c \) for all \( n \in \N \), then \( a \leq c\).
    \end{enumerate}
    \end{thm}
\end{tcolorbox}

\begin{enumerate}


    \item[(i)] \begin{proof}
We proceed by contradiction by assuming that \( a < 0 \). Suppose \( a_n \geq 0 \) and \( a_n \to a \). Let \( \epsilon  = | a |\) and suppose \( n \geq N \). Then
\[ | a_n - a | < | a | = -a.\]
But this means that \( a_N < 0\) which is a contradiction since \( a_N \geq 0\).
\end{proof}
    \item[(ii)]
        \begin{proof}
        We can ensure that the sequence \( b_n - a_n\) converges to \( b - a\) by the Algebraic Limit Theorem. Since \( b_n - a_n \geq 0\), we can use (i) to write \( b - a \geq 0\). Hence, \( a \leq b\).
        \end{proof}
    \item[(iii)]
        \begin{proof}
            Suppose there exists \( c \in \R  \) for which \( c \leq b_n\) for all \( n \in \N \). Suppose \( a_n  = c \) then using (ii) yields \( c \leq b\). Suppose \( a_n \leq c\) for all \( n \in \N \) then setting \( b_n = c \) and using (ii) again yields \( a \leq c\).
        \end{proof}
\end{enumerate}


\section{Exercises}


\subsection{Exercise 2.3.1} Let \( x_n \geq 0\) for all \( n \in \N \).

\begin{enumerate}
    \item[(a)] If \( (x_n ) \to 0 \), show that \( \sqrt{x_n} \to 0\).
        \begin{proof}
            Suppose \( x_n \geq 0 \) and \( x_n \to 0\). In order to show that \( \sqrt{x_n} \to 0\), it suffices to show that for every \( \epsilon> 0\), there exists an \( N \in \N \) such that for every \( n \geq N \) we have 
            \[ | \sqrt{x_n} - 0  | < \epsilon.\]
            Choose \( N \in \N\). Suppose \( x_n = 0 \) for all \( n \in \N \), then \( ( \sqrt{x_n}) = 0 \) for all \( n \geq N \) which means that \( ( \sqrt{x_n}) \to 0\). Suppose \( x_n > 0 \) for all \( n \in \N\), then observe that since \( (x_n) \to 0\)  and \( (x_n)\) bounded, we have 
            \begin{align*}
             | \sqrt{x_n} - 0 | &= | \sqrt{x_n} | \\
                            &= \Big| \frac{x_n}{ \sqrt{x_n}} \Big| \\
                            &=  \Big| \frac{x_n - 0}{ \sqrt{x_n}} \Big| \\ 
                            &=  \frac{|x_n - 0|}{ \sqrt{x_n}}  \\
                            &< \sqrt{M} \frac{ \epsilon }{ \sqrt{M} }  \\ 
                            &= \epsilon 
            \end{align*}
            Hence, it follows that \( ( \sqrt{x_n}) \to 0\).
        \end{proof}
    \item[(b)] If \( (x_n) \to x\), show that \( (\sqrt{x_n}) \to \sqrt{x}\).
        \begin{proof}
            Suppose that \( x_n \geq 0\) for all \( n \in \N \). Suppose \( (x_n) \to x \). We want to show that \( ( \sqrt{x_n}) \to x\). Suppose \( x_n = 0 \) and suppose \( N \in \N \) such that for every \( n \geq N \), then we have the first case above where \( x=0\) and \( ( \sqrt{x_n}) \to 0\). Now suppose \( x_n > 0 \) and choose \( N \in \N \) such that for every \( n \geq N \), then observe that since \( (x_n ) \to x \) and \( (x_n)\) is bounded by an integer \( M > 0 \), we have that 

        \begin{align*}
        |  \sqrt{x_n } - \sqrt{x}  | &= \Big| \frac{ x_n - x}{ \sqrt{x_n} + \sqrt{x}} \Big| \\
                          &= \frac{ | x_n - x |}{ |\sqrt{x_n} + \sqrt{x} | } \\
                          &< ( \sqrt{M} + \sqrt{x}) \frac{ \epsilon }{ (\sqrt{M} + \sqrt{x})} \\ 
                          &= \epsilon. 
        \end{align*}
        Hence, it follows that \( ( \sqrt{x_n}) \to \sqrt{x}\).
        \end{proof}
\end{enumerate}

\subsection{Exercise 2.3.2}
Using only Definition 2.2.3, prove that if \( (x_n) \to 2\), then 
\begin{enumerate}
    \item[(a)] \(  \Big(\frac{2x_n - 1}{ 3} \Big) \to 1\);
        \begin{proof}
            Suppose \( (x_n) \to 2\). Our goal is to show that property above. It suffices to show that for every \( \epsilon  > 0\), there exists \( N \in \N  \) such that for every  \( n \geq N  \), we have 
            \[ \Big| \frac{2x_n - 1}{3} - 1 \Big| < \epsilon. \]
        Choose \( N \in \N \) and suppose \( n \geq N \)
        \begin{align*}
        \Big| \frac{2x_n  - 1}{3} - 1 \Big| &= \Big| \frac{2x_n - 4}{3}  \Big| \\
                                    &= \Big| \frac{2}{3} (x_n - 2)\Big| \\
                                    &= \Big| \frac{2}{3} \Big| | x_n - 2 | \\ 
                                    &< \frac{2}{3} \cdot \frac{3 \epsilon }{2} \\
                                    &= \epsilon.
        \end{align*}
        Hence, it follows that 
        \[ \Big(  \frac{2x_n  - 1}{3}\Big) \to 1.\]
        \end{proof}
    \item[(b)] \( \Big(   \frac{1}{x_n}\Big) \to \frac{1}{2}\).
        \begin{proof}
        We want to show that for every \( \epsilon  > 0 \), there exists \( N \in \N \) such that for every \( n \geq N \), we have 
        \[ \Big| \frac{1}{x_n} - \frac{1}{2}\Big| < \epsilon.\]
        Choose \( N \in \N \) and assume \( n \geq N \). Since \( (x_n) \to 2\), we can write 
        \begin{align*}
        \Big| \frac{1}{x_n} - \frac{1}{2} \Big| &= \Big| \frac{2 - x_n}{2x_n} \Big| \\
                                                &= \frac{ | x_n - 2 |}{ 2|x_n |}. \tag{1} \\
        \end{align*}
        Since \( (x_n) \to 2\), we can set \( \epsilon  = 1\) so that we can lower bound the denominator of (1) using 
        \[ 2 - \epsilon < | x_n | \implies 1 < | x_n |. \]
        Then we can set \( N = \max \{ 1, \epsilon / 2 \}\) so that 
        \[ \frac{ | x_n - 2  |}{ 2| x_n |} < \frac{2 \epsilon }{2} = \epsilon \]
        which satisifes our desired property.
        \end{proof}

\end{enumerate}

\subsection{Exercise 2.3.3}
Show 
\begin{tcolorbox}
    \begin{thm}[Squeeze Theorem]
        If \( x_n \leq y_n \leq z_n\) for all \( n \in \N \), and if \( \lim x_n = \lim z_n = \ell\), then \( \lim y_n = \ell\).
    \end{thm}
\end{tcolorbox}
\begin{proof}
    Suppose \( x_n \leq y_n \leq z_n\) for all \( n \in \N \) and suppose \( \lim x_n = \lim z_n = \ell\). We want to show that \( \lim y_n = \ell\). By the Order Limit Theorem, we have \( x_n \leq y_n \leq z_n \) for all \( n \in \N \) implies that \( \ell \leq y_n \leq \ell\) for all \( n \in \N \). But this means that \( y_n = \ell\) for all \( n \in \N \). Hence, for every \( \epsilon  > 0\), there exists an \( N \in \N\) such that for every \( n \geq N\) 
    \[ | y_n - \ell | = | \ell - \ell | = 0 < \epsilon.\]
    Hence, it follows that \( \lim y_n = \ell\).
\end{proof}

\subsection{Exercise 2.3.4}
Let \( (a_n) \to 0\), and use the Algebraic Limit Theorem to compute each of the following limits (assuming the fractions are always defined).
\begin{enumerate}
    \item[(a)] \( \lim \Big( \frac{1 + 2a_n}{ 1 + 3a_n - 4a_n^2}   \Big) \)
        \begin{proof}[Solution]
        Let \( (a_n) \to 0\). Then
        \begin{align*}
            \lim \Big( \frac{ 1 + 2a_n}{ 1 + 3a_n - 4a_n^2 } \Big) &= \frac{ \lim (1 + 2a_n)}{ \lim (1 + 3a_n - 4a_n^2)} \\
                                 &= \frac{ \lim 1 + \lim (2a_n)}{ \lim 1 + \lim (3a_n) - \lim (4a_n^2)} \\
                                 &= \frac{ 1 + 2 \cdot 0}{ 1 + 3 \cdot 0 + 4 \cdot 0^2} \\
                                 &= 1.
        \end{align*}
        \end{proof}
    \item[(b)] \( \lim \Big( \frac{ (a_n + 2)^2 - 4}{a_n} \Big)\)
        \begin{proof}[Solution]
        Let \( (a_n) \to 0\). Then 
        \begin{align*}
          \lim  \Big( \frac{ (a_n + 2)^2 - 4}{ a_n} \Big)&= \lim \Big( \frac{a_n^2 + 4a_n}{a_n} \Big) \\
                                                 &= \lim \Big( a_n + 4 \Big) \\  
                                                 &= \lim a_n + \lim 4 \\ 
                                                 &= 0 + 4 \\ 
                                                 &= 4.
        \end{align*}
        \end{proof}
    \item[(c)] \( \lim \Big( \frac{ \frac{2}{a_n} + 3}{ \frac{1}{a_n} + 5} \Big)\).
        \begin{proof}[Solution]
        Let \( (a_n) \to 0\). Then 
        \begin{align*}
        \lim \Big( \frac{ \frac{2}{a_n} + 3}{ \frac{1}{a_n} + 5} \Big) &= \lim \Big( \frac{2 + 3a_n}{1 + 5a_n} \Big)  \\
                                                               &= \frac{ \lim 2 + \lim (3a_n)}{ \lim 1 + \lim (5a_n)} \\ 
                                                               &= \frac{2 + 3 \cdot 0}{ 1 + 5 \cdot 0}\\
                                                               &=2. 
        \end{align*}
        \end{proof}
\end{enumerate}

\subsection{Exercise 2.3.5}
Let \( (x_n)\) and \( (y_n)\) be given, and define \( (z_n)\) to be the "shuffled" sequence \[ (x_1, y_1, x_2, y_2 ,...,x_n, y_n).\] 
For the forwards direction, assume \( (z_n) \) is a convergent sequence. We want to show that \( \lim x_n = \lim y_n \). It suffices to show that given any \( \epsilon > 0\), there exists an \( N \in \N \) such that for any \( n \geq N \), we have 
\[ | x_n - y_n | < \epsilon.\]
Suppose \( (x_n) \to x\) and \( (y_n) \to y\), then we can write 
\begin{align*}
   | x_n - y_n  | &= | x_n - z_n + z_n - y_n |  \\
                  &< | x_n - z_n | + | z_n - y_n | \\  
                  &= | x_n - z + z - z_n | + | z_n - z + z - y_n | \\
                  &< | x_n - x | + | x - z_n | + | z_n - y | + | y - y_n |. \tag{1} 
\end{align*}
By definition, \( (z_n)\) is a shuffled sequence and convergent. Hence, \( z_n \to x \) and \( z_n \to y\). But by the uniqueness of limits, \( x = y\) so we have that 
\[ | x_n - y_n | < \frac{ \epsilon }{4} + \frac{ \epsilon }{4} +  \frac{ \epsilon }{4} + \frac{ \epsilon }{4} = \epsilon.\]
which means \( \lim (x_n - y_n) = \lim x_n - \lim y_n = 0\). 

Now for the backwards direction, assume \( \lim x_n = \lim y_n\). We want to show \( (z_n)\) converges i.e for every \( \epsilon> 0\), there exists \( N \in \N \) such that for every \( n \geq N \), we have 
\[ | z_n - z | < \epsilon.\]

\subsection{Exercise 2.3.6} 
Consider the sequence given by \( b_n = n - \sqrt{n^2 + 2n}\). Taking \( (1 / n ) \to 0\) as given, and using both the Algebraic Limit Theorem and the result in Exercise 2.3.1, show \( \lim b_n \) exists and find the value of the limit.

\begin{proof}
Consider the sequence given by \( b_n = n - \sqrt{n^2 + 2n}\). Assume \( (1 / n ) \to 0\) and \( \sqrt{x_n} \to \sqrt{x} \). Then taking the limit of \( b_n\), we have 
\begin{align*}
    \lim b_n &= \lim (n - \sqrt{n^2 + 2n}) \\
             &= \lim   \frac{-2n}{n + \sqrt{n^2 + 2n}} \\  
             &= \lim \frac{ -2}{ 1 + \sqrt{1 + 2 / n}} \\
             &= \frac{ \lim (-2) }{ \lim (1 + \sqrt{1 + 2 / n})} \\
             &= \frac{\lim(-2)}{\lim (1) + \lim (\sqrt{1 + 2 / n})} \\
             &= \frac{-2}{1 + 1 + 0} \tag{\((1/n) \to 0\), \( (\sqrt{x_n}) \to \sqrt{x} \) } \\
             &= -1.
\end{align*}
Hence, we have \( \lim b_n = -1 \). Now we can show that \( b_n\) does reach this limit. 

Let \( \epsilon > 0\). Then choose 
\[ N = \frac{2}{ \sqrt{ \frac{1 + \epsilon }{1 - \epsilon }} - 1}.\]
Then assumme \( n \geq N \). Our goal is to show that 
\[ | b_n + 1 | < \epsilon.\]
Then 
\begin{align*}
    &n >  \frac{2}{ \sqrt{ \frac{1 + \epsilon }{1 - \epsilon }} - 1} \\
    \implies &\sqrt{ \frac{1 + \epsilon }{1 - \epsilon }} -1  > \frac{2}{n}  \\  
\end{align*} Then we have 
\begin{align*}
  \sqrt{1+2/n}  &< \frac{1 + \epsilon }{1 - \epsilon } \\
  (1 - \epsilon )\sqrt{1 - 2/n} &< 1 + \epsilon \\   
  (1 - \epsilon )\sqrt{1 - 2/n} - 1 &< \epsilon .
\end{align*}
Then we get 
\begin{align*}
    - 1 + \sqrt{1 + 2 / n} < \epsilon ( 1 + \sqrt{ 1 + 2 / n})     \\
\end{align*}
and then 
\begin{align*}
    \frac{ - 1 + \sqrt{1 + 2/n}}{ 1 + \sqrt{ 1 + 2/n}} &< \epsilon \\ 
    \frac{ -2n}{ n + \sqrt{ n^2 + 2n}} + \frac{n + \sqrt{n^2 + 2n}}{ n + \sqrt{ n^2 + 2n}} &< \epsilon \\ 
    n - \sqrt{n^2 + 2n} + 1 &< \epsilon.
\end{align*}
Hence, it follows that \( | b_n + 1  | < \epsilon\).

\end{proof}

\subsection{Exercise 2.3.8}
Let \( (x_n) \to x \) and let \( p(x)\) be a polynomial.
\begin{enumerate}
    \item[(a)] Show \( p(x_n) \to p(x)\).
        \begin{proof}
            Let \( (x_n) \to x \) and let \( p(x)\) be a polynomial. Let 
            \[ p(x) = \sum_{i=0}^{m} a_i x^i\]
            and 
            \[ p(x_n) = \sum_{i=0}^{m} a_i x_n^{i}. \]
            Our goal is to show that for every \( \epsilon > 0\), there exists \( N \in \N \) such that for every \( n \geq N \)            we have 
            \[ | p(x_n) - p(x)| < \epsilon.\]
           Then by part (i) of the Algebraic Limit Theorem, we have
           \begin{align*}
            | p(x_n) - p(x) |  &= \Big| \sum_{i=0}^{m} a_i x_n^{i} - \sum_{i=0}^{ m} a_i x^{i}  \Big| \\
                               &= \Big| \sum_{i=0}^{m} a_i (x^i_n - x^i)  \Big| \\
                               &<  \sum_{ i = 0}^{ m} | a_ix^i_n - a_ix^i | \tag{T.I}\\     
                               &<  \sum_{i=0}^{m} \frac{ \epsilon }{ m}  \tag{\(x_n \to x\)} \\
                               &= \frac{ \epsilon }{m }  \cdot m \\ 
                               &= \epsilon. 
           \end{align*}
           Hence, we have \( p(x_n) \to p(x)\).

        \end{proof}
    \item[(b)] Find an example of a function \( f(x)\) and a convergent sequence \( (x_n) \to x\) where the sequence \( f(x_n)\) converges, but not to \( f(x)\).
        
\end{enumerate}

\subsection{Exercise 2.3.9}
\begin{enumerate}
    \item[(a)] Let \( (a_n)\) be a bounded (not necessarily convergent) sequence, and assume \( \lim b_n  = 0\). Show that \( \lim (a_n b_n) = 0\). Why are we not allowed to use the Algebraic Limit Theorem to prove this? 
        \begin{proof}
        Let \( (a_n)\) be a bounded but not necessarily convergent sequence, and assume \( \lim b_n = 0 \). We want to show that \( \lim (a_n b_n ) = 0\). It suffices to show that for every \( \epsilon  > 0\), there exists \( N \in \N\) such that for every \( n \geq N \), we have 
        \[ | a_nb_n - 0 | < \epsilon. \tag{1}\]
        Since \( (a_n)\) bounded, there exists an \( M > 0\) such that \( | a_n  | < M \). Starting with the left side of (1), choose \( N \in \N \) such that for every \( n \geq N \) 
        \begin{align*}
            | a_n b_n - 0 |&= | a_n | | b_n |\\
                           &< M \cdot \frac{ \epsilon }{M} \tag{\(b_n \to 0\)} \\ 
                           &= \epsilon. 
        \end{align*}
        Hence, it follows that \( \lim (a_n b_n) \to 0\). We cannot use the Algebraic Limit Theorem here because \( (a_n)\) does not necessarily have a defined limit even though it is bounded.
        \end{proof}
    \item[(b)] Can we conclude anything about the convergence of \( (a_nb_n)\) if we assume that \( (b_n)\) converges to some nonzero limit \( b \)?
        \begin{proof}[Solution]
        It would simply not converge.
        \end{proof}
    \item[(c)] Use (a) to prove Theorem 2.3.3, part (iii), for the case when \( a = 0\). 
        \begin{proof}
            Suppose \( a_n \to a\) where \( a = 0\) and \( b_n \to b\). Our goal is to show that \( \lim(a_n b_n) = 0\). Let \( \epsilon > 0 \), then choose \( N \in \N \) such that for every \( n \geq N \),  
        \begin{align*}
         | a_nb_n - 0| &< | a_n | | b_n |   \\
                       &< \frac{ \epsilon }{ M } \cdot M  \tag{\( a_n \to 0, b_n \to b\)} \\ 
                       &< \epsilon   
        \end{align*}
        Hence, it follows that \( \lim (a_nb_n) = 0\).
        \end{proof}
\end{enumerate}

\subsection{Exercise 2.3.10} Consider the following list of conjectures. Provide a short proof that are true and a counterexample for any that are false.
\begin{enumerate}
    \item[(a)] If \( \lim (a_n - b_n) = 0 \), then \( \lim a_n = \lim b_n\). 
        \begin{proof}[Counterexample]
        Suppose \( a_n = \frac{n}{2n+1}\) and \( b_n = \frac{n}{2n+5}\). We have \( \lim a_n = \lim b_n \) but \( \lim (a_n - b_n ) \neq 0\).
        \end{proof}
    \item[(b)] If \( (b_n) \to b\), then \( | b_n | \to | b |\).
        \begin{proof}
        Let \( \epsilon > 0\). Consider \( | |b_n| - |b|| \). Assume \( n \geq N \) then 
        \begin{align*}
           | |b_n| - |b|| &< | b_n - b| < \epsilon   \
        \end{align*}
        by reverse triangle inequality and \( (b_n) \to b\).
        \end{proof}
    \item[(c)] If \( (a_n) \to a\) and \( (b_n - a_n) \to 0\), then \( (b_n) \to a\).
        \begin{proof}
        Assume \( (a_n) \to a\) and \( (b_n - a_n) \to 0\). Let \( \epsilon  > 0\). By assumption, 
        \begin{align*}
            | a_n - a | &< \frac{ \epsilon }{2}  \text{,   \space  }  n \geq N_1\\
            | b_n - a_n | &< \frac{ \epsilon }{2} \text{, \space } n \geq N_2.
        \end{align*}
        Hence, choose \(  N = \max \{ N_1, N_2 \} \) such that 
        \begin{align*}
        | b_n - a |&= | b_n - a_n + a_n - a  | \\
                   &< | b_n - a_n | + | a_n - a | \\
                   &< \frac{ \epsilon }{2} + \frac{ \epsilon }{2} \\ 
                   &= \epsilon. 
        \end{align*}
        Hence, \( (b_n) \to a\).
        \end{proof}
    \item[(d)]If \( (a_n) \to a\) and \( | b_n - b | \leq a_n \) for all \( n \in \N \), then \( (b_n) \to b\).
        \begin{proof}
        Let \( \epsilon > 0\). Choose \( N \) so that \( a_n \to 0 \). Then consider \( |b_n - b |\) and observe that
        \begin{align*}
        | b_n - b| &\leq a_n < \epsilon, 
        \end{align*}
        Hence, it follows that \( (b_n ) \to b\).
        \end{proof}
\end{enumerate}




