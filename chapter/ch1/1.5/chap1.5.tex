


\chapter{Cardinality}

\subsection{Correspondence}

\begin{tcolorbox}
\begin{defn}
    A function \( f: A \to B \) is \textit{one-to-one} if \( a_1 \neq a_2 \) in \( A \) implies that 
    \( f(a_1) \neq f(a_2) \) in \( B \). The function \( f \) is \textit{onto} if, given any \( b \in B \), there exists an element \( a \in A \) for which \( f(a) = b\).
\end{defn}
\end{tcolorbox}

An equivalent defintion for a function to be one-to-one is the following:

\begin{tcolorbox}
\begin{defn}
    A function \( f: A \to B \) is \textit{one-to-one} if \( f(a_1) \neq f(a_2) \) implies that \( a_1 = a_2 \).
\end{defn}
\end{tcolorbox}
    
A function that is both one-to-one and onto is said to be bijective. Meaning that we have a one-to-one correspodence between the sets \( A \) and \( B \). Another way to explain a function being injective is to say that no two elements from \( A \) can map to the same element in \( B \) ( think of the function \( x^2\)). And a function being onto can be explained as every element in \( A \) has to be mapped to an element in \( B \).

From an algebraic perspective, we can denote a function being bijective to mean the same thing as two sets having the same cardinality i.e we can say that 

\begin{tcolorbox}
\begin{defn}
Two sets \( A \) and \( B \) have the same cardinality if there exists \( f: A \to B \) that is both one-to-one and onto. We can denote this symbolically as \( A \sim B\)
\end{defn}
\end{tcolorbox}

\begin{ex}
Some examples of bijective maps are
\begin{enumerate}
    \item Let the following map \(f: \N \to \mathbf{E} \) be defined as \( f(n) = 2n \). We can see that \( \N \sim \mathbf{E}\). It's true that \( \mathbf{E}\) is indeed a subset of \( \N \), but do not conclude that it is a smaller set than \( \N \) since they have the same cardinality or isomorphic to each other.
    \item We can show this again. This time let us have a map \( f: \N \to \Z \) such that 
        \[ f(n) = \begin{cases}  (n-1)/2 \text{ if } n \text{ is odd.}       \\
                                 -n/2 \text{ if } n \text{ is even.}
                                                \end{cases}\]
We have that \( \N \sim \Z \) indeed.
\end{enumerate}
\end{ex}


\subsection{Countable Sets}

\begin{tcolorbox}
\begin{defn}
A set \( A \) is \textit{countable} if \( \N \sim A \). An infinite set that is not countable is called an \textit{uncountable set}.
\end{defn}
\end{tcolorbox}

\begin{tcolorbox}
\begin{thm}
Let \( \Q, \R \). Then 

\begin{itemize}
    \item The set \( \Q \) is countable. 
    \item The set \( \R \) is uncountable.
\end{itemize}


\end{thm} 
\end{tcolorbox}


\begin{proof}
\begin{enumerate}
\item Suppose we define \( A_n \) to be split into two sets. When \( n = 1 \), define \( A_n \) to be 
\[ A_1 = \{  0 \}\] and define \( A_n \) when \( n \geq 2 \) as 
\[ A_n =   \Big\{ \pm \frac{p}{q} : \text{ where } p,q \in \N \text{ are in lowest terms with } p + q = n   \Big\}\]
We can observe here that for every \( n \in \N \) we can find every element of \( \Q \) exactly once in the sets we have defined. So we can conclude that our map is onto. Since we designed our sets so that each rational numer appears once and the fact that for \( n =1 \) and \( n \geq 2 \) produces two disjoint sets, we can see that our map is also one-to-one. 
\item We can prove that second statement of theorem by contradiction. Assume for the sake of contradiction that there exists a \textit{one-to-one} and \textit{onto} function where \( f: \N \to \R \). Letting \( x_1 = f(1)\) and \( x_2 = f(2)\) and so on, then we can enumerate each element of \( \R \) i.e 
    \[ \R = \{ x_1, x_2, x_3, ... \}.\]
    Using the Nested Interval Property, we will now produce a real number that is not in this set. Let \( I_n \) be a closed interval which does not contain  \( x_n \) but contains \( x_{n+1}\). Furthermore, \( I_{n+1}\) is contained within \( I_n \). Note that within \( I_n \) there are two sets which are disjoint and \( x_{n+1}\) can be in either one of these sets. Now consider the following intersection \( \cap_{n=1}^{\infty} I_n \). Using our construction that every \(x_n \not\in I_n \), then we can say that
    \begin{align*} \bigcap_{n=1 }^{\infty} I_n = \emptyset. \end{align*}
But this is a contradiction because the nested interval property asserts that this intersection is nonempty meaning that every \( x \in \R \) is contained in the above set. Hence, we cannot emumerate every single element \( x_n  \) of \( \R \). Therefore, \( \R \) is an \textit{uncountable} set.
\end{enumerate}
\end{proof}

This gives us three insights: 

\begin{enumerate}
    \item The smallest type of infinite set is the countable set.
    \item We can create another set by deleting or inserting elements into it. 
    \item Anything smaller than a countable set is either finite or countable. 
\end{enumerate}

We can create \( \R \) by taking the union of \( \Q \) and \( \mathbb{I} \). Since \( \R \) is not countable and \( \Q \) is, this would mean that the set of irrational numbers \( \mathbb{I}\) would be uncountable. This tells us that \( \mathbb{I}\) is a bigger subset of \( \R \) than \( \Q \). 

We can summarize these results in the follow two theorems: 


\begin{tcolorbox}
\begin{thm}
    If \( A \subseteq B \) and \( B \) is \textit{countable}, then \( A \) is either countable or finite. 
\end{thm}
\end{tcolorbox}

\begin{tcolorbox}
\begin{thm}
\begin{enumerate}
    \item If \( A_1, A_2,... A_n\) are each countable sets, then the union of 
        \[ A_1 \cup A_2 \cup ... \cup A_m \] is countable.
    \item If \( A_n \) is a countable set for each \( n \in \N \), the \( \bigcup_{n=1}^{\infty}A_n \) is countable. 
\end{enumerate}
\end{thm}
\end{tcolorbox}


\subsection{Exercises}
\subsubsection{Exercise 1.5.1}
Finish the following proof for Theorem 1.5.7.
\begin{proof}
Assume \( B \) is a countable set. So there exists a map \( f: \N \to B \) such that \( f \) is surjective and injective. Let \( A \subseteq B \) be an infinite subset of \( B \). We want to show that \( A \) is countable. That is, \( A \) is both 
\begin{enumerate}
    \item injective 
    \item surjective. 
\end{enumerate}
Let \( n_1 = \min \{ n \in \N: f(n) \in A  \}\). Let \( g: \N \to A  \) be the map defined by 
\[ g(1) = f(n_1).\]
To show injectivity of \( g\), we proceed via induction on the index \( i \in \N \). Let the base case be \( i = 2 \). Then suppose \( g(1) = g(2) \). By definition of \( g \) and injectivity of \( f \), we have that 
\begin{align*}
g(1)&= g(2) \\ 
f(n_1) &= f(n_2) \\
n_1 &= n_2.
\end{align*}
But this means that \( n_2 =\min  \{  n \in \N : f(n) \in A  \}\). Hence, \( g \) is injective. Now for the inductive step, assume this holds for every \( 1 \leq i \leq k - 1 \). We want to show that this holds for \( i = k \). Suppose that 
\[ g(1) = g(k).\]
By defintion of \( g \) and injectivity of \( f \), we have that 
\begin{align*}
f(n_1)&=f(n_k) \\
n_1 &= n_k.
\end{align*}
But this also means that \( n_k = \min \{ n \in \N : f(n) \in \N \} \). Hence, \( g \) is injective. 

Now we want to show that \( g \) is surjective. Note that we have 
\[ g(i) = A \cap \{ f(n_1), f(n_2), f(n_3), ..., f(n_k) \}.\]
Then by definition of \( g \), we have that \( g(i) = f(n_i) \). Since \( f \) is surjective, there exists some \( b \in B \) such that \( f(n_i) = b \). But since \( n_i = \{ n_i \in \N : f(n_i) \in A  \}\), we have that \( f(n_i) \in A \) so \( g \) is surjective as well. Hence, we have that \( g \) is both injective and surjective which means that \( \N \sim A \). Therefore, \( A \) is countable.  

\end{proof} 




\subsubsection{Exercise 1.5.2}

Review the proof of Theorem 1.5.6, part (ii) showing that \( \R \) is uncountable, and then find the flaw in the following erroneous proof that \( \Q \) is uncountable. 

The proposition is: 
\( \Q \) is uncountable. 

\begin{proof}
Assume for contradiction that \( \Q \) is countable. Thus we can write \( \Q = \{  r_1, r_2, r_3 \}\) and as before, construct a nested sequence of closed intervals with \( r_n \not\in I_n \). Our construction implies \( \bigcap_{n=1}^{\infty} I_n = \emptyset \) while the nested interval property implies that this intersection is nonempty. This contradiction implies \( \Q \) must therefore be uncountable.
\end{proof}
\begin{proof}[Thoughts]
I think the main issue with this proof is when the author assumed that the set of rationals are closed. Since \( \Q \) contains irrantional numbers within each subset of the \( \Q \) as well as real numbers, \( \Q \) cannot be closed.  Hence, we cannot apply the nested interval property here.  
\end{proof}


\subsubsection{Exercise 1.5.3}

Prove theorem 1.5.8



\begin{tcolorbox}
\begin{thm}
\begin{enumerate}
    \item If \( A_1, A_2,... A_n\) are each countable sets, then the union of 
        \[ A_1 \cup A_2 \cup ... \cup A_m \] is countable.
    \item If \( A_n \) is a countable set for each \( n \in \N \), the \( \bigcup_{n=1}^{\infty}A_n \) is countable. 
\end{enumerate}
\end{thm}
\end{tcolorbox}

\begin{enumerate}
    \item First, prove statement (i) for two countable sets, \( A_1 \) and \( A_2 \).
    \begin{proof}
    Suppose \( A_1 \) and \( A_2 \) are countable sets. Then \( \N \sim A_1 \) and \( \N \sim A_2 \). Furthermore, we have that the maps \( f: \N \to A_1 \) and \( g: \N \to A_2 \) are bijective. Our goal is to show the union \( A_1 \cup A_2 \) is also countable i.e we need to show that 
    the map \( h: \N \to A_1 \cup A_2 \) is bijective. Before we proceed, let us replace \( A_2 \) with the following set \( B_2 \) defined as 
    \[ B_2 = A_2 \setminus A_1 = \{ h(n) \in A_2 : h(n) \not\in A_1 \}.\]
Now our following map is \( h: \N \to A_1 \cup B_2\) (this is equivalent to \( A_1 \cup A_2 \))and define it as follows
\begin{align*}
  h(n) = \begin{cases}
          f(k) \text{ if } n=2k              \\
          g(k)    \text{ if } n=2k+1 \in B_2
        \end{cases}
\end{align*}
Suppose we have \( n_1, n_2 \in \N \) and \( h(n_1) = h(n_2) \). Since \( f \) and \( g \) are injective, we have 

\begin{align*}
   h(n_1) &= h(n_2) \\
   f(n_1) &= f(n_2) \\ 
   n_1 &= n_2.
\end{align*}
This shows that \( h \) is injective (the same process can be applied to \( g \) when \( h \in B_2 \)). Note that \( A_1 \cap B_2 = \emptyset \) because otherwise \( h \) would not be well defined. Now we need to show that \( h \) is surjective. Since \( f \) and \( g \) are surjective, there exists either \( x \in A_1 \) or \( x \in B_2 \) such that \( h(n) = f(n) = x \) or \( h(n) = g(n) = x \). Hence, we have that \( h \) is surjective. 
Since \( h \) is a bijective map, we now have that \( \N  \sim A_1 \cup B_2 \).

Suppose we use induction on the index \( i \in \N \). Since we have already proven the base case for two countable sets, let us assume \( A_1, A_2, ..., A_k \) are all countable sets such that for \( i \leq k - 1 \), the union \( A_1 \cup A_2 ... \cup A_{k-1}\) is countable. Let's set \( A' = A_1 \cup A_2 ... \cup A_{k-1} \). Our goal is to show that the union \( A' \cup A_k \) is countable. Let's define the map \(h: \N \to A' \cup B' \) such that 
\[ B' = A_k \setminus A' = \{ h(n) \in A_k : x \not\in A' \}.\]
and 
\begin{align*}
  h(n) = \begin{cases}
          f(k) \text{ if } n = 2k              \\
          g(k)    \text{ if } n = 2k+1
        \end{cases}
\end{align*}
Let \( n_1, n_2 \in \N \). Since \( A' \) and \( A_k \) are countable sets, we have that  
\begin{align*}
   h(n_1) &= h(n_2) \\
   f(n_1) &= f(n_2) \\ 
   n_1 &= n_2.
\end{align*}
Hence, \( h \) is injective. Now we want to show that \( h \) is surjective. If either \( h(n) \in A'\) or \( h(n) \in A_k  \), then since \( f: \N \to A' \) and \( g: \N \to A_k \) are surjective functions, we have that there exists \( x \in A_k \) or \( x \in A' \) such that \( h(n) = x \). Hence, \( h \) is surjective as well.
Since \( h \) is now a bijective function, we conclude that the union \( A_1 \cup A_2 ... \cup A_{k}\) is countable. 
\[ \]
\end{proof}
    

\item 
Explain why induction cannot be used to prove part (1) of Theorem 1.5.8 from part (2)
\begin{proof}[Solution]
We cannot use induction on part (2) of theorem 1.5.8 because the index itself \( n \in \N \) is infinite and induction only works only finite \( n \). 
\end{proof}


\end{enumerate}






\subsubsection{Proof of the second part of theorem}

\begin{proof}
    Let \( \{ S_n \}_{n \in \N} \) be a sequence of countable sets. Define the union 
    \[ S = \bigcup_{n \in \N } S_n. \]
    for all \( n \in \N \). Assume each \( S_n \) is disjoint. Otherwise, let \( S_1\) such that for each \( n \geq 1 \), define 
    \[ S'_{n+1} = S_{n+1} \setminus S_{n} = \{  x \in S_{n+1} : x \notin S_{n}   \}.\]
    This is to ensure that our following map is well-defined.
    let \( F_n \) denote the set of all injections from \( S_n \to \N \)
    Let \( \varphi: S \to \N \times \N \) be the map that is defined by 
    \[ \varphi(x) = (n, f_n(x))\] 
    where \( n \in \N \) smallest guranteed by the Well-Ordering Principle. Since each \(f_n \) is an injection, it follows that \( \varphi \) is also an injection.   
Since \( \N \times \N\) is countable, there exists an injection \( \alpha: \N \times \N \to \N \). Composing the two functions \( \varphi \) and \( \alpha \), we have that \( \alpha \circ \varphi: S \to \N \) is an injection. Since \( \N \times \N \sim \N \), we know that \( \alpha \) is also surjective. Hence, the composition \( \alpha \circ \varphi \) is also surjective. Therefore, we have that \( S \) is countable. 



\end{proof}



\subsubsection{Exercise 1.5.4}

\begin{enumerate}
    \item[(a)] Show \( (a,b ) \sim \R \) for any interval \( (a,b) \).  
    \begin{proof}
    Let \( (a,b) \) any interval. Then define the function \( f: (a,b) \to \R \) as 
    \[ f(x) = x^2 \]
    Our objective is to show that \( f \) is injective and surjective. To show that \( f \) is injective, we need to let \( x_1, x_2 \in (a,b) \). Then suppose 
    \[ f(x_1) = f(x_2).\]
    Then we have that 
    \begin{align*}
    f(x_1)&= f(x_2) \\
    x_1^2 &= x_2^2 \\ 
    x_1 &= x_2.
    \end{align*}
    This shows that \( f \) is injective. Now we want to show that \( f \) is surjective. Then there exists \( \sqrt{y} \in (a,b)\). Let 
    \[ x = \sqrt{y}.\]
    Then we have that 
    \begin{align*}
    x^2 &= y \\
    f(x) &= y.
    \end{align*}
    Hence, \( f \) is surjective. Since \( f: (a,b) \to \R \) is a bijective function, we have that \( (a,b) \sim \R \). 
    \end{proof}
    \item[(b)] Show that an unbounded interval like \( (a, \infty ) = \{ x: x > a \}\) has the same cardinality as \( \R \) as well.
    \begin{proof}
     Let \( (0, \infty ) = \{ x: x > 0 \}\). Our goal is to show that \( (a, \infty ) \sim \R \). To show this, we need to show the map \(f: (a, \infty ) \to \R \) is bijective. Define \( f \)as the following:
     \[ f(x) = \ln(x).\]
     Then suppose \( f(x_1) = f(x_2) \) for \( x_1, x_2 \in (a, \infty)\). Then 
    \begin{align*}
     \ln(x_1) &= \ln(x_2) \\ 
     x_1 &= x_2 
    \end{align*}
    Hence, we have that \( f\) is an injective function. Now we want to show that \( f \) is surjective. Then let \( e^{y}  = x \in (0,\infty)\). Then taking the natural log of both sides, we have that \( \ln(x) = y \). Hence, we have that \( f \) is a surjective function. Since \( f \) is a bijective function, we know that \( (0, \infty ) \sim \R \).  
    \end{proof}
    \item[(c)] Using open intervals makes it more covenient to produce the required 1-1, onto functions, but it is not really necessary. Show that \( [0,1) \sim (0,1) \) by exhibiting a 1-1 onto function between the two sets.
    \begin{proof}
        We want to show that \( [0,1) \sim (0,1) \). Define the map \( f: [0,1) \to (0,1) \) as
        \[ f(x) = \frac{1}{x-1} \]
        Our goal is to show that this map is bijective. Hence, we need to show that this map is both injective and surjective.

        To show that \( f \) is injective. Suppose \( f(x_1) = f(x_2) \) for \( x_1, x_2 \in [0,1)\). Then we have that 
        \begin{align*}
        f(x_1)&=f(x_2) \\
        \frac{1}{x_1-1} &= \frac{1}{x_2-1} \\
        x_1 - 1 &= x_2 - 1 \\
        x_1 &= x_2.
        \end{align*}
        Hence, \( f \) is injective. 

        To show that \( f \) is surjective, suppose we have \( x-1 = \frac{1}{y} \). Then 
        \[ y = \frac{1}{x-1}.\] But we have that \( f(x) = \frac{1}{x-1}\) so we have 
        \[ f(x) = \frac{1}{x-1} = y. \] Hence, \( f \) is surjective.

        Since \( f \) is bijective, we have that \( [0,1) \sim (0,1)\).
    \end{proof}
\end{enumerate}





\subsubsection{Exercise 1.5.5}

\begin{enumerate}
    \item[(a)] Why is \( A \sim A \) for every set \( A \)? 
    \begin{proof}[Solution]
    \( A \sim A \) because \( A \) is a bijection onto itself (same elements map to the same elements of the same set). 
    \end{proof}
    \item[(b)] Given sets \( A \) and \( B \), explain why \( A \sim B \) is equivalent to asserting \( B \sim A \).
    \begin{proof}[Solution]
    If \( A \sim B \), then the map \( f: A \to B \) is a bijection. Meaning we can map unique eleemnts from \( A \) to unique elements to \( B \). Since there is unique mapping of elements from \( A \to B \) thenn we would expect to see the same thing when we map the same elements from \( B \to A \). 
    \end{proof}
    \item[(c)] For three sets \( A, B, \) and \( C \), show that \( A \sim B \) and \( B \sim C \) implies \( A \sim C \). These three properties are what is meant by saying that \( \sim \) is an \textit{equivalence relation}.
    \begin{proof}
     Suppose we have three sets \( A, B, \) and \( C \). Suppose \( A \sim B \) and \( B \sim C \)
     then we have two maps \( f: A \to B \) and \( g: B \to C \) that are bijective. Composing the two functions we get \( g \circ f : A \to C \). We want to show that this mapping is also bijective. Let \( x_1, x_2 \in A \) then suppose \( g \circ f (x_1) = g \circ f (x_2) \). By definition of composition, we have 
     \begin{align*}
     g(f(x_1))&= g(f(x_2)) \\
     f(x_1) &= f(x_2)  \tag{ \( g \) is injective } \\ 
     x_1 &= x_2.       \tag{ \( f \) is injective }
     \end{align*}
     Hence, \( g \circ f \) is an injective function. Now we want to show that \( g \circ f \) is a surjective mapping. Since \( f \) is surjective, there exists a \( y \in B \) such that \( f(x) = y \). Since \( g \) is also surjective, there exists a \( z \in C \) such that \( g(y) = c\). Hence, we have that \( g(f(x)) = c \) which means \( g \circ f \) is a surjective mapping. Therefore, \( A \sim C\).   
 \end{proof}



 \subsubsection{Exercise 1.5.11}[Shroder-Bernstein Theorem]
 Assume there exists an injective function \( f: X \to Y \) and another injective function \( g: Y \to X \). Show that \( X \sim Y \). The strategy is to partition \( X \) and \( Y \) into components
 \begin{align*}
 X &=  A \cup A' \\
 Y &= B \cup B'
\end{align*}
 with \( A \cup A' = \emptyset \) and \( B \cup B' = \emptyset\), in such a way that \( f \) maps \( A \) onto \( B \), and \( g \) maps \( B' \) onto \( A' \)
\begin{enumerate}
    \item Explain how achieving this would lead to a proof that \( X \sim Y\).
    \begin{proof}[Solution]
    Taking disjoint sets prevents the problem of an element from either map mapping to two elements onto it's image. Thus, allowing us to have a well-defined function. Having two injective maps also would lead to the \( X \sim Y \) because composing these two functions would allow us take a unique mapping from one element from each other.    
    \end{proof}
    \item Set \( A_1 = X \setminus  g(Y) = \{ x \in X: x \notin g(Y) \}\) (what happens if \( A_1 = \emptyset? \)) and inductively define a sequence of sets by letting \( A_{n+1} = g(f(A_n))\). Show that \( \{ A_{n} : n \in \N  \}\) is a pairwise disjoint collection of subsets of \( X \), while \( \{ f(A_n) : n \in \N  \}\) is similar collection in \( Y \).
    \begin{proof}
        We set proceed by induction on \( n \in \N \) and let \( P(n)\) be the statement that 
        \( \{  A_n : n \in \N  \}\) and \( \{ f(A_n) : n \in \N  \}\) are pairwise disjoint.
         Define 
        \[ A_{n+1} = g(f(A_n))\] and 
        for each \( n \in \N \). Note that \( g(f(A_{n+1})) = A_{n+2}\).

        Let our base case be \( n = 1 \). Then \( A_2 = g(f(A_1))\). By definition of \( A_1\), we have that \( x \in X \) but not in \( g(Y)\). If \( x \notin g(Y) \) then \( x \notin g(f(A_1)) \) as well. Hence, we have that \( A_1 \cap A_2 = \emptyset\)        
Now assume \( P(n)\) holds for \( n \leq k-1 \). Define  
        \[ A_{n+1} = A_{n+1} \setminus g(f(A_{n+1})) = \{  x \in A_{n+1} : x \notin g(f(A_{n+1})) \}\]
        Since \(A_n \) is pairwise disjoint for each \( n \leq k-1 \), we also have that

        \[ A_{k-1} = A_{k-1} \setminus g(f(A_{k-1})) = \{  x \in A_{k-1} : x \notin g(f(A_{k-1})) \}.\]
        But notice that \( g(f(A_{k-1})) = A_k\) which tells us that  \( A_{k-1} \cap A_{k} \) are also pairwise disjoint. 
        Thus, \( A_n \) for each \( n \in \N \) is pairwise disjoint.

    \end{proof}
    \item Let \( A = \bigcup_{n=1}^{ \infty } A_n \) and \( B  = \bigcup_{n=1}^{\infty } f(A_n) \). Show that \( f \) maps \( A \) onto \( B \).

    \begin{proof}
        
    \end{proof}
    \item Let \( A' = X \setminus A \) and \( B' = Y \setminus B  \). Show \( g \) maps \( B' \) onto \( A'\)
        \begin{proof}
        
        \end{proof}
\end{enumerate} 





\end{enumerate}






